\documentclass[11pt]{article}
\usepackage[T1]{fontenc}
\usepackage{amsmath}
\usepackage{amssymb}
\usepackage{graphicx}
\usepackage[margin=3cm]{geometry}
\usepackage{listings}
\usepackage{ulem}
\usepackage{tikz}

\setlength{\parskip}{1mm}

% \includegraphics [height=50mm] {bild.png}

\setlength{\headheight}{40pt}

\begin{document}
	\title{Microcontroller programming \\
		Automatic Brightness Adjustment\\
		TE663 Uppsala Universitet VT2015}
	\date{}
	\author{}
	\author{Lars Haulin \texttt{lars.haulin.5967@student.uu.se}\\
		Robin Keller \texttt{robin.keller.4499@student.uu.se}\\
		Viktor Nordmark \texttt{viktor.nordmark.0315@student.uu.se}}
	\maketitle
	
   \abstract{In this project, a control system for lights was created.
   The sensor and dimmer communication via radio communication and keep a constant
   level of illumination, regardless of external disturbances such as a sudden
   cloud blocking external sunlight.}

	\section{Concept}
	
	Our idea is to control light using a dimmer and a light sensor.

   The project concept was chosen to make maximal use of our teams experience
   in power electronics, automatic control and network algorithms.

	The sensor and dimmer communicate with radio over unidirectional network link.
	There is also a possibility to add a user interface over the network to control
	the brightness.
	
	Should the room get brighter or darker from sunlight or other illumination,
	the lights should be adjusted to compensate for this.
	
	\section{Primary goals}

   The primary goals were
	
	\begin{itemize}
		\item To control the brightness of a 230V light bulb with a microcontroller.
		\item To measure the amount of light in a room with a microcontroller.
		\item To communicate data between the microcontrollers with a radio module.
		\item To make the system capable to compensate for changes in illumination.
	\end{itemize}
	
	\section{Secondary goals}
   Additionally, we put up some secondary goals:
	\begin{itemize}
		\item To control the system with a user interface node, also containing a microcontroller.
		\item To have several lamps and sensors.
		\item To automate calibration i.e., what lamp affects what sensor.
	\end{itemize}

   \section{Hardware design}
   As we had previous experience in our team of the nRF24L01+ radio module from
   Nordic Semiconductor, it was chosen as a base for the radio communication.

   We chose to dim the lights with a triac.
   
   \section{Radio Module nRF24L01+}
   The nRF24L01+ is a radio module which uses the world wide 2.4GHz ISM frequency band. It can operate with an air data rates of 250kbps, 1Mpbs and 2Mbps. To communicate with the radio module the Serial Peripheral Interface used. For every module you can configure a frequency channel and an address. The data which will be sent is shifted to an internal data pipe. With the send-command the data pipe is interpreted as a packet and will send to the previous configured address.
   
   You are able to configure the \textit{Enhanced ShockBurst\texttrademark} which is a special data link layer. It supports automatic packets handling, especially auto acknowledge of packets.
   A communication which uses this auto acknowledge is only working between two nodes which use the same address.
   
   \section{Data Link Layer Library for AVR and RaspberryPi}
   The library for the radio module nRF24L01+ was previous written by Robin Keller and will be reused in this project. Therefore the configuration for the AVR ATmega328 was added to the library.
   The library does not pass the complete functionality of the radio module to the user. Unicast and broadcast functionality is implemented. The module identification for unicast communication takes place by an unique address per module. For broadcast communication the address \texttt{0xFF} is reserved.
   
   After initializing the radio module the module is in receiving mode. Thus it will receive packets to its own address or the broadcast address \texttt{0xFF}. The auto acknowledge feature of the nRF24L01+ is automatically enabled by library.
   
   The module can be initialize by calling the function \texttt{DLL\_init} where you need to set the address, auto retransmission count, RF channel and the air data rate.
   
   For sending a packet call the \texttt{DLL\_send} function with the send type, receiver address, data and the length of data. In the send type you can specify if you want to request an acknowledge packet or not. The maximum length of one packet is 32 Byte.
   
   To receive a packet call the function \texttt{DLL\_receive}, which returns \texttt{0} if there is no received data in the module FIFO. To avoid polling this function, there is the variable \texttt{DLL\_irq\_rx\_counter} which indicated that the radio module received a packet. This variable is set in an interrupt from the radio module library.
   
     \section{Power electronics}
     The project utilizes 230 Volt directly from a standard power socket. In order to handle this potential risk of electric shock as safe as possible we have chosen an optocoupler to separate the high voltage circuits from the lower voltage, control circuits. 
The voltage was controlled from the microcontroller via a triac that actuated the on/off commands in the phase. In order to know where in the sinus cycle we currently are, we utilized a zero crossing detection in the form of  an optocoupler that sent a signal every time the phase went from positive to negative or vice versa. As the frequency in the Swedish power grid is 50 Hertz, the zero crossing will occur with a frequency of 100 Hertz. 
When the voltage crosses the zero point, the triac remains of and the proper wait time is calculated. When the accurate time has been achieved, the triac turns on and remains on until the sinus wave crosses the zero point once more.
The longer the wait time is, the less power the light source recieves.
The dimming circuit was based on drawings found online. (See drawing in appedix)

\section{Phase control theory}

The brightness of the light source is controlled by limiting how much of the sinus wave the light source recieves by adjusting the time when the triac is actuated. 

	\begin{figure}[ht!]
    \centering
    \includegraphics[width=0.8\textwidth]{Sinuswave.png}
    \caption{Sinus wave}
    \label{fig:Sinus_wave}
\end{figure}

During the first period($t_{1}$) of the sinus wave the triac remains in off-state, waiting until the desired time has passed for the triac to fire and thus controll the brightness. When the time has passed, the microcontroller applies voltage to the triac's gate input just long enough($t_2$) for the triac to fire and let the remaining part of the sinus wave reach the light source.

\section{Dimming software}

Every time the alternating voltage passes through the zero point, the program jumps to an interrupt routine where the desired wait time is calculated and the triac output is activated. Since the voltage is rectified the interrupt routine is called with a frequency of 100Hz on the rising edge of the wave.







\end{document}
