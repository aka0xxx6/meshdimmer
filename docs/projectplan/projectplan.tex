\documentclass[11pt]{article}
\usepackage[T1]{fontenc}
\usepackage{amsmath}
\usepackage{amssymb}
\usepackage{graphicx}
\usepackage[margin=3cm]{geometry}
\usepackage{listings}
\usepackage{ulem}
\usepackage{tikz}

\setlength{\parskip}{1mm}

% \includegraphics [height=50mm] {bild.png}

\setlength{\headheight}{40pt}

\begin{document}
	\title{Microcontroller programming \\
		Project Plan\\
		TE663 Uppsala Universitet VT2015}
	\date{}
	\author{}
	\author{Lars Haulin \texttt{lars.haulin.5967@student.uu.se}\\
		Robin Keller \texttt{robin.keller.4499@student.uu.se}\\
		Viktor Nordmark \texttt{viktor.nordmark.0315@student.uu.se}}
	\maketitle
	
	\section{Concept}
	
	Our idea is to control lights using dimmers and light sensors.
	The sensors and dimmers communicate with radio over a mesh network.
	There is also a possibility to add a user interface over the network to control
	the brightness.
	
	Should the room get brighter or darker from sunlight or other illumination,
	the lights should be adjusted to compensate for this.
	
	\section{Primary goals}
	
	\begin{itemize}
		\item To control the brightness of a 230V light bulb with a microcontroller.
		\item To measure the amount of light in a room with a microcontroller.
		\item To communicate data between the microcontrollers over a mesh network.
		\item To make the system capable to compensate for changes in illumination.
		\item To control the system with a user interface node, also containing
		a microcontroller.
	\end{itemize}
	
	\section{Secondary goals}
	\begin{itemize}
		\item To have several lamps and sensors
		\item To automate calibration i.e., what lamp affects what sensor
	\end{itemize}
	
	
\end{document}